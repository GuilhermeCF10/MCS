\begin{titlepage}
    \thispagestyle{empty} % Remove números de página
    \setstretch{1.5} % Espaçamento entre linhas, certifique-se de que o pacote setspace está incluído em document.tex

    \begin{center}
        \textbf{\Large RESUMO}
    \end{center}

    \vspace{1cm} % Espaço vertical

    % \noindent % Garante que o primeiro parágrafo não seja indentado
    % Santos, Leonardo Pereira. \textit{Simulação em Scilab em um sistema massa-mola-amortecedor}, 2023. Projeto de conclusão da disciplina Modelagem e Controle de Sistemas - Instituto Politécnico, Universidade do Estado do Rio de Janeiro, Nova Friburgo, 2023.

    % \vspace{1cm} % Espaço vertical

    % \noindent
    % Neste trabalho, veremos de que modo a ferramenta e linguagem Scilab consegue auxiliar na modelagem de um sistema físico, desde a parte da criação do modelo matemático até a interpretação do comportamento do mesmo por meio de gráficos.

    % \vspace{0.5cm} % Espaço vertical

    % \noindent
    % Palavras-chave: Sistema massa-mola-amortecedor, função de transferência, scilab.
\end{titlepage}
