%% 3.1.9 Resumo em língua portuguesa
%% Elemento obrigatório (Figura 14).
%% Consiste na apresentação sucinta dos pontos relevantes do texto,
%% em um único parágrafo. O resumo deve conter entre 150 e 500 pala-
%% vras e fornecer uma visão rápida e clara dos objetivos, da metodologia,
%% dos resultados e das conclusões do trabalho. Na elaboração do resumo,
%% deve-se usar o verbo na voz ativa, na terceira pessoa do singular.

%% Fonte -> TNR ou Arial, corpo 12.
%% A palavra RESUMO deve aparecer em letras maiúsculas
%% e em negrito.
%% O uso de itálico é permitido em palavras estrangeiras.
%% O uso de letras maiúsculas nas palavras-chave
%% restringe-se ao início da palavra, em nomes próprios
%% e siglas, se for o caso.

%% Alinhamento -> A palavra RESUMO deve estar localizada na margem
%% superior da folha e centralizada, e a referência, alinhada
%% à margem esquerda;
%% O alinhamento é justificado para o texto do resumo,
%% que inicia com parágrafo, e para as palavras-chave.

%% Espaçamento -> A palavra RESUMO deve ser separada da referência por
%% duas linhas em branco de 1,5;
%% Espaço 1 na referência e no resumo e, nas palavras-
%% chave, espaço 1,5.

%% Formato do papel,
%% orientação e margens -> Conforme especificado na seção 1.1.

%% Pontuação -> As palavras-chave devem ser separadas por ponto e
%% terminadas por ponto.

\begin{resumo}
\begin{SingleSpace}

\noindent
\begin{flushleft}
\entradaAutor{}. \textit{\imprimirtitulo}. \the\year. \pageref{LastPage} f. Trabalho de Conclusão de Curso (Graduação em Engenharia de Computação) - Instituto Politécnico, Universidade do Estado do Rio de Janeiro, Nova Friburgo, \the\year.
\end{flushleft}
\vspace{\onelineskip}

\setlength{\parindent}{1.3cm}

\textit{Consiste na apresentação sucinta dos \textbf{pontos relevantes} do texto, em um único parágrafo. O resumo deve conter\textbf{ entre 150 e 500 palavras} e fornecer uma\textbf{ visão rápida e clara dos objetivos}, da metodologia, dos resultados e das conclusões do trabalho. Na elaboração do resumo, deve-se usar o \textbf{verbo na voz ativa}, na terceira pessoa do singular.>
}

\vspace{0.5cm}
O envelhecimento da população no Brasil gera desafios de saúde, incluindo aumento do Alzheimer e risco de fraturas por quedas, afetando a taxa de mortalidade dos idosos. A detecção de crises epilépticas é crucial, dada a falta de consciência dos pacientes, dificultando o ajuste adequado da medicação. A identificação de ocorrências como quedas, ataques epilépticos e parâmetros físicos do usuário é crucial para ajustes no tratamento. Além disso, localizar os usuários também é essencial. \textbf{Nesse contexto}, com o objetivo de reduzir as sequelas causadas após quedas, principalmente em idosos, um sistema de monitoramento e assistência médica torna-se fundamental para abordar esses desafios e melhorar a qualidade de vida da população.  \textbf{Neste projeto}, foi desenvolvido um protótipo que consiste na criação de um sistema ......\textbf{Para atingir esses objetivos, foi desenvolvido} este projeto no qual foi utilizado o microcontrolador ESP32, associado a um acelerometro, um oximetro e medidor de batimentos cardiacos, e um módulo GPS. Os software referente a montagem foi todo desenvolvido dentro da plataforma Arduino IDE, enquanto que a parte refente ao setup foi desenvolvida na linguagem PHP. O protótipo foi validado e o \textbf{resultado geral do projeto} foi satisfatório, pois algumas melhorias em relação a medição da pressão arterial e temperatura corporal.
 

\vspace{\onelineskip}
\noindent Palavras-chave:  assistência médica; bracelete; ESP32; PHP; saúde; XAMPP.

\end{SingleSpace}
\end{resumo}